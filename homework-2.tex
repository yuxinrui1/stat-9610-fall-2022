% (C) Brett Klamer - MIT - http://opensource.org/licenses/MIT
% Please contact me if you find any errors or make improvements
% Contact details at brettklamer.com

\documentclass[11pt,letterpaper,english,oneside]{article} % article class is a standard class
%==============================================================================
%Load Packages
%==============================================================================
\usepackage[left=1in,right=1in,top=1in,bottom=1in]{geometry} % easy page margins
\usepackage[utf8]{inputenc} % editor uses utf-8 encoding
\usepackage[T1]{fontenc} % T1 font pdf output
\usepackage{lmodern} % Latin modern roman font
\usepackage{bm, bbm} % bold and blackboard bold math symbols
\usepackage{amsmath, amsfonts, amssymb, amsthm} % math packages
\usepackage[final]{microtype} % better microtypography
\usepackage{graphicx} % for easier grahics handling
\usepackage[hidelinks, colorlinks=true, linkcolor = blue, urlcolor = blue]{hyperref} % to create hyperlinks
\usepackage{float} % tells floats to stay [H]ere!
\usepackage{enumitem} % nice lists
\usepackage{fancyhdr} % nice headers
\usepackage{caption}  % to control figure and table captions
\usepackage{booktabs} % to create nice tables

\captionsetup{width=0.9\textwidth, justification = raggedright}

%==============================================================================
% Enter name and homework title here
%==============================================================================
\author{Name}
\title{STAT 9610: Homework 2}
\date{Due October 11, 2022 at 10:00am}

%==============================================================================
% Put title and author in PDF properties
%==============================================================================
\makeatletter % change interpretation of @
\hypersetup{pdftitle={\@title},pdfauthor={\@author}}


%==============================================================================
% Header settings
%==============================================================================
\pagestyle{fancy} % turns on fancy header styles
\fancyhf{} % clear all header and footer fields
\makeatletter
\lhead{\@author} % left header
\chead{\@title} % center header
\makeatother
\rhead{Page \thepage} % right header
\setlength{\headheight}{13.6pt} % fixes minor warning
\makeatother % change back interpretation of @

%==============================================================================
% List spacing
%==============================================================================
\setlist[itemize]{parsep=0em} % fix itemize spacing
\setlist[enumerate]{parsep=0em} % fix enumerate spacing

%==============================================================================
% Float spacing (changes spacing of tables, graphs, etc)
%==============================================================================
%\setlength{\textfloatsep}{3pt}
%\setlength{\intextsep}{3pt}

%==============================================================================
% Define Problem and Solution Environments
%==============================================================================
\theoremstyle{definition} % use definition's look
\newtheorem{problem}{Problem}
\newtheorem{solution}{Solution}
\newenvironment{prob}{\clearpage \begin{problem}\hspace{0pt}}{\end{problem}}
\newenvironment{sol}{\begin{solution}\hspace{0pt}}{\end{solution}}

\begin{document}

\maketitle

\section{Instructions}

\paragraph{Setup.} Clone this repository and open \verb|homework-2.tex| in your LaTeX editor. Use this document as a starting point for your writeup, adding your solutions between \verb|\begin{sol}| and \verb|\end{sol}|. Add R code for problem $i$ in \verb|problem-i.R| (rather than in your LaTeX report), saving your figures and tables to the \verb|figures-and-tables| folder for LaTeX import. 

\paragraph{Resources.}

Consult the \href{https://katsevich-teaching.github.io/stat-9610-fall-2022/assets/getting-started.pdf}{getting started guide} if you need to brush up on R, LaTeX, or Git, the \href{https://katsevich-teaching.github.io/stat-9610-fall-2022/assets/preparing-reports.pdf}{preparing reports guide} for guidelines on presentation quality, the \href{https://github.com/stat-9610-fall-2022/sample-homework-stat-9610}{sample homework} for an example of a completed homework repository, and \href{https://hmc-cs-131-spring2020.github.io/howtos/assignments.html}{this webpage} for more detailed instructions on using GitHub and Gradescope to complete and submit homework.

\paragraph{Programming.}

The \verb|tidyverse| paradigm for data manipulation (\verb|dplyr|) and plotting (\verb|ggplot2|) is required; points will be deducted for using base R. 

\paragraph{Grading.} Each sub-part of each problem will be worth 3 points: 0 points for no solution or completely wrong solution; 1 point for some progress; 2 points for a mostly correct solution; 3 points for a complete and correct solution modulo small flaws. The presentation quality of the solution for each problem (see the \href{https://katsevich-teaching.github.io/stat-9610-fall-2022/assets/preparing-reports.pdf}{preparing reports guide}) will be evaluated out of an additional 3 points.

\paragraph{Submission.} Compile your LaTeX report to PDF and commit your work. Then, push your work to GitHub. Finally, submit your GitHub repository to \href{https://www.gradescope.com/courses/423692}{Gradescope}.

\paragraph{Collaboration.} The collaboration policy is as stated on the Syllabus:

\begin{quote}
``Students are permitted to work together on homework assignments, but solutions must be written up and submitted individually. Students must disclose any sources of assistance they received; furthermore, they are prohibited from verbatim copying from any source and from consulting solutions to problems that may be available online and/or from past iterations of the course.''
\end{quote}

\noindent In accordance with this policy, \\

\noindent \textcolor{red}{\textit{Please list anyone you discussed this homework with:}} 

\clearpage

\begin{prob} \label{prob:likelihood}\textbf{Likelihood inference in linear regression.} \\

    \noindent Let's consider the usual linear regression setup. Given a full-rank $n \times p$ model matrix $\bm X$, a coefficient vector $\bm \beta \in \mathbb R^p$, and a noise variance $\sigma^2 > 0$, suppose
    \begin{equation}
    \bm y = \bm X \bm \beta + \bm \epsilon, \quad \bm \epsilon \sim N(0, \sigma^2 \bm I_n).
    \label{eq:linear-model}
    \end{equation}
    The goal of this problem is to connect linear regression inference with classical likelihood-based inference (below is a quick refresher).
    
    \begin{enumerate}
    \item[(a)] For the sake of simplicity, let's start by assuming $\sigma^2$ is known. Under the fixed-design model, why does the linear regression model~\eqref{eq:linear-model} not fit into the classical inferential setup~\eqref{eq:iid-sampling}? Write the linear model in as close a form as possible to~\eqref{eq:iid-sampling}.
    
    \item[(b)] Continue assuming that $\sigma^2$ is known. Why does the Fisher information~\eqref{eq:fisher-info} not immediately make sense for the linear regression model? Propose and compute an analog to this quantity, and using this quantity exhibit a result analogous to the asymptotic normality~\eqref{eq:asymptotic-normality}.
    
    \item[(c)] Now assume that neither $\bm{\beta}$ nor $\sigma^2$ is known. Derive the maximum likelihood estimates for $(\bm \beta, \sigma^2)$. How do these compare to the estimates $(\bm{\widehat \beta}, \widehat \sigma^2)$ discussed in class?
    
    \item[(d)] Continuing to assume that neither $\bm{\beta}$ nor $\sigma^2$ is known, consider the null hypothesis $H_0: \bm{\beta}_S = \bm 0$ for some $S \subseteq \{1, \dots, p\}$. Write this hypothesis in the form~\eqref{eq:general-null-hypothesis}, and derive the likelihood ratio test for this hypothesis. Discuss the connection of this test with the $F$-test.
    
    \end{enumerate}
    
    \noindent\fbox{\begin{minipage}{\textwidth}
    \paragraph{Refresher on likelihood inference.} In classical likelihood inference, we have observations 
    \begin{equation}
    y_i \overset{\text{i.i.d.}}\sim p_{\bm \theta}, \quad i = 1, \dots, n
    \label{eq:iid-sampling}
    \end{equation}
    from some model parameterized by a vector $\bm \theta \in \Theta \subseteq \mathbb R^d$. Under regularity conditions, the maximum likelihood estimate $\bm{\widehat \theta}_n$ is known to converge to a normal distribution centered at its true value:
    \begin{equation}
    \sqrt n(\bm{\widehat \theta}_n - \bm \theta) \overset d \rightarrow N(0, \bm I(\bm \theta)^{-1}),
    \label{eq:asymptotic-normality}
    \end{equation}
    where 
    \begin{equation}
    \bm I(\bm \theta) \equiv -\mathbb E_{\bm \theta}\left[\frac{\partial^2}{\partial \bm \theta^2} \log p_{\bm \theta}(y) \right]
    \label{eq:fisher-info}
    \end{equation}
    is the per-observation Fisher information matrix.
    Furthermore, an optimal test of the null hypothesis 
    \begin{equation}
    H_0: \bm \theta \in \Theta_0 \quad \text{versus} \quad H_1: \bm \theta \in \Theta_1 
    \label{eq:general-null-hypothesis}
    \end{equation}
    for some $\Theta_0 \subseteq \Theta_1 \subseteq \Theta$ is the likelihood ratio test based on the test statistic
    \begin{equation}
    \Lambda = \frac{\max_{\bm \theta \in \Theta_1}\prod_{i = 1}^n p_{\bm \theta}(y_i)}{\max_{\bm \theta \in \Theta_0}\prod_{i = 1}^n p_{\bm \theta}(y_i)}.
    \end{equation}
    Under $H_0$, we have the convergence 
    \begin{equation}
    2 \log \Lambda \overset d \rightarrow \chi^2_k, \quad \text{where} \quad k \equiv \text{dim}(\Theta_1) - \text{dim}(\Theta_0).
    \end{equation}
    \end{minipage}
    }
    
    \end{prob}
    
    \begin{sol}
    \end{sol}
    
    \begin{prob} \textbf{Relationships among $t$-tests, $F$-tests, and $R^2$.} \\
    
    \noindent Consider the linear regression model~\eqref{eq:linear-model}, such that $\bm x_{*,0} = \bm 1_n$ is an intercept term.
    
    \begin{itemize}
    \item[(a)] Relate the $R^2$ of the linear regression to the $F$-statistic for a certain hypothesis test. What is the corresponding null hypothesis? What is the null distribution of the $F$-statistic? Are $R^2$ and $F$ positively or negative related, and why does this make sense?
    
    \item[(b)] Use the relationship found in part (a) to simulate the null distribution of the $R^2$ by repeatedly sampling from an $F$ distribution (via \verb|rf|). Fix $n = 100$ and try $p \in \{2, 25, 50, 75, 99\}$. Comment on these null distributions, how they change as a function of $p$, and why. 
    
    \item[(c)] Consider the null hypothesis $H_0: \beta_j = 0$, which can be tested using either a $t$-test or an $F$-test. Write down the corresponding $t$ and $F$ statistics, and prove that the latter is the square of the former. 
    
    \item[(d)] Now suppose we are interested in testing the null hypothesis $H_0: \bm \beta_{-0} = \bm 0$. One way of going about this is to start with the usual test statistic $t(\bm c)$ for the null hypothesis $H_0: \bm c^T \bm \beta_{-0} = 0$, and then maximize over all $\bm c \in \mathbb R^{p-1}$:
    \begin{equation}
    t_{\max} \equiv \max_{\bm c \in \mathbb R^{p-1}} t(\bm c).
    \end{equation}
    What is the null distribution of $t_{\max}^2$? What $F$-statistic is $t_{\max}^2$ equivalent to? How does the null distribution of $t_{\max}^2$ compare to that of $t(\bm c)^2$?
    
    \end{itemize}
    
    \end{prob}
    \begin{sol}
    \end{sol}
    
    \begin{prob} \label{prob:data}\textbf{Case study: Violent crime.}
    
    \noindent The \texttt{Statewide\_crime.tsv} file contains information on the number of violent crimes and murders for each U.S. state in a given year, as well as three socioeconomic indicators: percent living in metropolitan areas, high school graduation rate, and poverty rate (Table~\ref{tab:crime-data}).

    \begin{table}[!h]

\caption{\label{tab:crime-data}The first five rows of the crime data.}
\centering
\begin{tabular}[t]{lrrrrr}
\toprule
STATE & Violent & Murder & Metro & HighSchool & Poverty\\
\midrule
AK & 593 & 6 & 65.6 & 90.2 & 8.0\\
AL & 430 & 7 & 55.4 & 82.4 & 13.7\\
AR & 456 & 6 & 52.5 & 79.2 & 12.1\\
AZ & 513 & 8 & 88.2 & 84.4 & 11.9\\
CA & 579 & 7 & 94.4 & 81.3 & 10.5\\
\bottomrule
\end{tabular}
\end{table}


    \noindent The goal of this problem is to study the relationship between the three socioeconomic indicators and the per capita violent crime rate.
    
    \begin{itemize}
    \item[(a)] These data contain the total number of violent crimes per state, but it is more meaningful to model violent crime rate per capita. To this end, go online to find a table of current populations for each state. Augment \verb|crime_data| with a new variable called \verb|Pop| with this population information (see \verb|left_join()| from the \verb|dplyr| package) and create a new variable called \verb|CrimeRate| defined as \verb|CrimeRate = Violent/Pop| (see \verb|mutate()| from the \verb|dplyr| package).
    
    \item[(b)] Explore the variation and covariation among the variables \verb|CrimeRate|, \verb|Metro|, \verb|HighSchool|, \verb|Poverty| with the help of visualizations and summary statistics.
    
    \item[(c)] Construct linear model based hypothesis tests and confidence intervals associated with the relationship between \verb|CrimeRate| and the three socioeconomic variables, including any relevant tables or plots in your LaTeX report. Discuss the results in technical terms.
    
    \item[(d)] Discuss your interpretation of the results from part (c) in language that a policymaker could comprehend, including any caveats or limitations of the analysis. Comment on what other data you might want to gather for a more sophisticated analysis of violent crime.
    
    \end{itemize}
    
    \end{prob}
    \begin{sol}
    \end{sol}

\end{document}